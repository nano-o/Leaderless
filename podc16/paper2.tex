\documentclass[11,draft]{article}
\usepackage[english]{babel}
\usepackage{mathtools}
\usepackage{amsfonts}
\usepackage{amssymb}
\usepackage{amsthm}

\usepackage{hyperref}
\usepackage{cleveref}
\usepackage{comment}
\hypersetup{pdftex,colorlinks=true,allcolors=blue}
\usepackage{enumitem}
\setlist{leftmargin=14pt}
\usepackage{hypcap}
\usepackage[url=true,doi=false,isbn=false,backend=biber,style=numeric,maxnames=2]{biblatex}
\usepackage{color}
\usepackage{ifdraft}

\usepackage[compact]{titlesec}
\setlength{\belowcaptionskip}{-10pt}
\setlength{\abovecaptionskip}{5pt}

\usepackage{amsmath}
\usepackage{tabularx}
\usepackage{algorithm}
\usepackage{algpseudocode}
\usepackage{varwidth}
\usepackage{verbatim}
\usepackage{subfigure}
\addbibresource{managed.bib}

\algrenewcommand\algorithmicprocedure{}
\algtext*{EndProcedure}% Remove text
\algtext*{EndWhile}% Remove text
\algtext*{EndFor}% Remove text
\algtext*{EndIf}% Remove text
\algblockdefx[NAME]{START}{END}%
   [3][Unknown]{#2 #1#3}%
   {}
\algtext*{END}% Remove text

%\setcounter{tocdepth}{2}
%\usepackage{pdfpages}
%\usepackage{wrapfig}
%\makeatletter
%\newcommand{\setword}[2]{%
%    \phantomsection%
%    #1\def\@currentlabel{\unexpanded{#1}}\label{#2}%
%}
%\makeatother

\newcommand{\todo}[1]{\ifdraft{\textcolor{red}{TODO\@: #1}}{}}

\DeclarePairedDelimiter\asetx{\{}{\}}
\newcommand{\aset}[1]{\asetx*{#1}}
\newcommand{\Aset}[1]{\bigl\{#1\bigr\}}

\DeclarePairedDelimiter\aseqx{\langle}{\rangle}
\newcommand{\aseq}[1]{\aseqx*{#1}}
\newcommand{\Aseq}[1]{\big\langle#1\big\rangle}

\DeclarePairedDelimiter\areqx{[}{]}
\newcommand{\arec}[1]{\areqx{#1}}
\newcommand{\Arec}[1]{\bigl[#1\bigr]}

\newcommand{\quo}[1]{\text{``#1''}}
\newcommand{\opn}[1]{\operatorname{#1}}

\usepackage{geometry}
\geometry{left=1in,right=1in,top=1in,bottom=1in}

\makeatletter
\def\operator@font{\sf}
\makeatother

\newtheorem{mylemma}{Lemma}
\crefname{mylemma}{lemma}{lemmas}
\newtheorem{mydef}{Definition}
\crefname{mydef}{definition}{definitions}

\makeatletter
\def\thm@space@setup{%
    \thm@preskip=\parskip \thm@postskip=0pt
}
\makeatother

\makeatletter
\newenvironment{proof*}[1][\proofname]{\par
    \pushQED{\qed}%
    \normalfont \partopsep=\z@skip \topsep=\z@skip
    \trivlist
\item[\hskip\labelsep
        \itshape
    #1\@addpunct{.}]\ignorespaces
}{%
    \popQED\endtrivlist\@endpefalse
}
\makeatother


\title{\vspace{-0.8in}Brief Announcement: A Family of Leaderless Generalized-Consensus Algorithms}

\begin{comment}
\author{
{\small Author 1}\\
{\small ECE, Virginia Tech}\\
{\small email}
\and
{\small Author 2}\\
{\small ECE, Virginia Tech}\\
{\small email}
\and
{\small Author 3}\\
{\small ECE, Virginia Tech}\\
{\small email}
}
\end{comment}

%\begin{comment}

\author{
Giuliano Losa\\
ECE, Virginia Tech\\
giuli4no@vt.edu
\and
Sebastiano Peluso\\
ECE, Virginia Tech\\
peluso@vt.edu
\and
Binoy Ravindran\\
ECE, Virginia Tech\\
binoy@vt.edu
}
%\end{comment}

\date{}

\begin{document}

\maketitle

%\begin{titlepage}


\title{Brief Announcement: A Family of Leaderless Generalized-Consensus Algorithms}

\author{
Author 1\\
ECE, Virginia Tech\\
email
\and
Author 2\\
ECE, Virginia Tech\\
email
\and
Author 3\\
ECE, Virginia Tech\\
email
}


\date{}

\maketitle \thispagestyle{empty}

%=========================================================================
%  Abstract
%=========================================================================

\begin{abstract}
Abstract here.
\end{abstract}

\bigskip

\centerline{{\bf Keywords}: Key1; Key2; Key3; Key4.}

\end{titlepage}

\newpage

\vspace{-0.5cm}
\section{Introduction}

Many agreement algorithms inspired by Paxos and MultiPaxos~\cite{lamport2001paxos} rely on the existence of a unique leader process which enforces an ordering on commands during fault-free periods.
In practice, the unique leader can easily become a performance bottleneck for the system.
Therefore, recently proposed agreement algorithms like Mencius~\cite{MaoJunqueiraMarzullo08MenciusBuildingEfficientReplicatedStateMachine}, EPaxos~\cite{MoraruAndersenKaminsky13ThereIsMoreConsensusEgalitarianParliaments}, or Alvin~\cite{TurcuETAL14BeGeneralDontGiveUpConsistency} strive to increase performance by avoiding the use of a unique leader.
Mencius rotates the leader among the processes but its structure remains close to MultiPaxos. 
However, EPaxos and Alvin are based on a novel algorithmic idea first introduced in EPaxos.

We show that the core idea underlying EPaxos can be captured in a generic algorithm that uses two abstractions: a dependency-set algorithm, which suggests dependencies for commands, and a map-agreement algorithm, which ensures that, for each submitted command, processes agree on a dependency set. Both abstractions lend themselves well to implementations that avoid the use of a unique leader process.
Our generic algorithm gives rise to a family of algorithms whose members are obtained by using a concrete agreement algorithm and a concrete dependency-set algorithm. 
%Note that any generalized consensus algorithm can be used to implement (2), but only a weaker problem than generalized consensus needs to be solved because no ordering among proposal is required.

On top of enabling modular correctness proofs of algorithms like EPaxos, we expect that the modular structure of our generic leaderless algorithm will allow a principled theoretical and empirical evaluation of the trade-offs that can be achieved by different implementations of our two abstractions, as well as optimizations to their orchestration.
We have formalized our generic leaderless algorithm in TLA+ and have thoroughly model-checked our safety claims with the TLC model-checker.

\section{Leaderless Generalized Consensus Algorithms}

We consider a set of processes $P$ subject to crash-stop faults and communicating by message-passing in an asynchronous network. Processes in $P$ must solve the Generalized Consensus problem~\cite{Lamport05GeneralizeConsensus}. Each process receives proposals for commands of the form $\opn{Propose}(c)$, and must produce responses of the form $\opn{Decide}(\sigma)$, where $\sigma$ is a c-struct. 
The sequence of calls observed must satisfy the non-triviality, consistency, stability, and liveness properties of generalized consensus.
We use the same notation and refer the reader to~\cite{Lamport05GeneralizeConsensus} for a thourough discussion of c-structs and generalized consensus. Briefly, the c-structs are a set, including $\bot$, with an operator $\bullet$ for appending commands to a c-struct. Intuitively, a c-struct represents a set of sequences that are all equivalent up to the ordering of commutative commands.
We say that two commands \textit{commute} when $\sigma\bullet c_1\bullet c_2 =  \sigma\bullet c_2\bullet c_1 $, for every c-struct $\sigma$.
C-structs are partially ordered: $\sigma_1\sqsubseteq \sigma_2$ iff there exists a sequence of commands $s$ such that $\sigma_2 = \sigma_1\bullet s$ and $\bot\sqsubseteq \sigma$ for any
$\sigma$. Two c-structs are compatible when they have a
common upper bound. A c-struct contains a command $c$ when it is of the form $\bot\bullet cs$ where $c$ appears in the sequence of commands $cs$. The consistency property of generalized consensus requires that any two decided c-structs be compatible.

Our leaderless generalized consensus algorithm orchestrates a dependency-set algorithm and map-agreement algorithm as shown in \cref{fig:deps-set-algo3}.
A process $p$ maintains two local variables $m_p$ (a map from commands to sets of commands) and $\sigma_p$ (a c-struct). Upon  receiving a command $c$, $p$ proceeds in three phases. First, it calls a \textit{dependency-set} algorithm (\Cref{dep-algo}) to determine a set of commands $\mathcal{D}$ that the command $c$ may depend on. Second, $p$ proposes the mapping  $c\mapsto \mathcal{D}$ to a \textit{map-agreement} algorithm (\Cref{map-algo}); when the map-agreement algorithm
commits a mapping $c\mapsto \mathcal{D}'$ for $c$, $p$ inserts the mapping $\left[c \mapsto \mathcal{D}'\right]$ in $m_p$, and we say that $\mathcal{D}'$ is the dependency set of $c$.
We say that $c$ is executable in $m_p$ when all its transitive dependencies are known. %no command $c'$ for which no mapping has been committed is reachable from $c$ by recursively following committed mappings. 
Finally, when $c$ is executable in $m_p$, then $p$ may execute $c$ by deciding the c-struct obtained by \emph{locally} running a \emph{graph-processing} algorithm described in \Cref{abstract-algo}.
Any other process can run the three phases independently of $p$ and eventually execute the command, possibly despite a crash of $p$.
%We show that the leaderless  generalized consensus algorithm guarantees the safety and liveness properties of generalized consensus when its underlying dependency-set and map-agreement algorithms satisfy their safety and liveness properties.

%In the next sections we present our specification of leaderless generalized consensus algorithms, which is parameterized over a \textit{dependency-set} algorithm (\Cref{dep-algo}) and a \textit{map-agreement} algorithm (\Cref{map-algo}). 


\subsection{Computing Dependency-Sets}
\label{dep-algo}

A dependency-set algorithm exposes at each process $p$ the input interface $\opn{announce}_p\left( c \right)$, to announce a command $c$, the output interface $\opn{suggest}_p \left( c,\mathcal{D}\right)$, to suggest a set of dependencies $\mathcal{D}$ for $c$, and an input interface $\opn{commit}\left( c,\mathcal{D} \right)$,  to observe which command is committed by the other abstraction.
Moreover, a dependency-set algorithm must ensure that:
\begin{itemize}[noitemsep,nolistsep]
    \item[-] Safety: (\setword{S1}{dep-valid}), for any suggestion $\opn{suggest}_p \left( c,\mathcal{D}\right)$, every command in $\{c\}\cup \mathcal{D}$ must have been announced before; (\setword{S2}{suggestion-agreement}), for any two suggestions $\opn{suggest}_p \left( c_1,\mathcal{D}_1\right)$ and $\opn{suggest}_q \left( c_2,\mathcal{D}_2\right)$, if $c_1$ and $c_2$ do not commute then either $c_1\in \mathcal{D}_2$ or $c_2 \in \mathcal{D}_1$.
%    \item If $c$ was announced by $p$, then eventually a dependency set is suggested for $c$ to $p$.
    \item[-] Liveness: (\setword{L1}{eventual-suggestion}) If a suggestion $\opn{suggest}_p\left( c \right)$ was made, then eventually a suggestion $\opn{suggest}\left( c,\mathcal{D} \right)_p$ is made, and (\setword{L2}{re-suggest}), if a suggestion $\opn{suggest}_p(c,\mathcal{D})$ has been made and no input $\opn{commit}_q\left( c,\mathcal{D'} \right)$ is observed, then eventually a new suggestion $\opn{suggest}_r\left( c,\mathcal{D''} \right)$ for $r\neq p$ is made.
\end{itemize}

\begin{figure*}
\centering
\subfigure[Leaderless generalized consensus implementation.]{
\noindent\fbox{%
	\begin{varwidth}{\dimexpr\linewidth-2\fboxsep-2\fboxrule\relax}
 
    
		\begin{algorithmic}[1]
			{\scriptsize
			\START[$\opn{init}_p$]{{\bf upon}}{$()$}
			\State $m_p\gets \emptyset$
			\State $\sigma_p\gets \bot$
			\END
			\line(1,0){80}
			\START[$\opn{GC-propose}_p$]{{\bf upon}}{$(c)$}
			\State {\bf call} $\opn{announce}_p\left( c \right)$
			\END
			\line(1,0){80}
			\START[$\opn{suggest}_p$]{{\bf upon}}{$(c,\mathcal{D})$}
			\State  $\opn{propose}_p( c,\mathcal{D})$ 			
			\END
			\line(1,0){80}
			\START[$\opn{commit}_p$]{{\bf upon}}{$(c,\mathcal{D})$}
			\State  $m_p \gets m_p \oplus \left[ c \mapsto \mathcal{D}\right]$	
			\State $\sigma_p \gets  \opn{c-struct}\left(m_p\right)$
			\State $\opn{GC-decide}_p(\sigma_p)$		
			\END
			}
		\end{algorithmic}
	\end{varwidth}%
}
\label{fig:deps-set-algo3}
}
\hspace*{0.1in}\subfigure[Example of dependency-set algorithm implementation - part 1.]{
\noindent\fbox{%
	\begin{varwidth}{\dimexpr\linewidth-2\fboxsep-2\fboxrule\relax}

		\begin{algorithmic}[1]
			{\scriptsize
			\START[\texttt{init$_p$}]{{\bf upon}}{$()$} \label{alg:init}
			\State $cmds_p\gets\emptyset$ \label{alg:init-cmds}
			\State $\forall c,$ $deps_p[c]\gets \emptyset$ \label{alg:init-deps}
			\State $\forall c,$ $heard_p[c]\gets \emptyset$ \label{alg:init-heard}
			\State $\forall c,$ $suggested_p[c]\gets false$ \label{alg:init-suggested}
			\END
			\line(1,0){108}
			\START[\texttt{announce$_p$}]{{\bf upon}}{$(c)$} \label{alg:announce}
			\State $cmds_p\gets cmds_p\cup\{c\}$ \label{alg:add-command}
			\State $\forall q\in P$ {\bf send} $\langle \texttt{cmd}$ $|$ $c \rangle$ {\bf to} $q$\label{alg:send-new-cmd}
			\END
			}
		\end{algorithmic}
	\end{varwidth}%
}
\label{fig:deps-set-algo1}
}	
\hspace*{0.1in}\subfigure[Example of dependency-set algorithm implementation - part 2.]{
\noindent\fbox{%
	\begin{varwidth}{\dimexpr\linewidth-2\fboxsep-2\fboxrule\relax}
	
		\begin{algorithmic}[1]
			\makeatletter
			\setcounter{ALG@line}{8}
			\makeatother
			{\scriptsize
			\START[$\langle \texttt{cmd}$ $|$ $c \rangle$]{{\bf upon receive}}{ {\bf from} $q$} \label{alg:receive-new-cmd}
			\State $\mathcal{D} \gets \{ d : d \in cmds_p \land$\label{alg:update-d}
			\Statex \hspace{2cm}$\land d\neq c \land d \asymp c \}$
			\State $cmds_p\gets cmds_p\cup\{c\}$\label{alg:update-cmds-p}
			\State {\bf send} $\langle \texttt{deps}$ $|$ $c,\mathcal{D} \rangle$ {\bf to} $q$\label{alg:send-deps}
			\END
			\line(1,0){125}
			\START[$\langle \texttt{deps}$ $|$ $c,\mathcal{D} \rangle$]{{\bf upon receive}}{ {\bf from} $q$} \label{alg:receive-deps}
			\State $deps_p[c]\gets deps_p[c]\cup \mathcal{D}$\label{alg:update-deps-p}
			\State $heard_p[c]\gets heard_p[c]\cup \{q\}$
			\END
			\line(1,0){125}
			\START[]{{\bf upon}}{$\exists c$ $:$ $ | heard_p[c] | \geq \mathcal{Q}$ $\land$} \label{alg:receive-deps}
			\Statex \hspace{1cm}$\land$ $suggested_p[c]=false$
			\State $suggested_p[c]\gets true$
			\State {\bf call} $\texttt{suggest}_p(c,deps_p[c])$ \label{alg:call-suggest}
			\END
			}
		\end{algorithmic}
	\end{varwidth}%
}
\label{fig:deps-set-algo2}
}
\caption{Leaderless generalized consensus, and example of dependency-set algorithm implementation.}
\label{fig:deps-set-algo}
\end{figure*}

%\todo{Plain english version below. Which is better?}

%A dependency-set computation algorithm allows processes to announce a command $c$ and to learn about a suggested dependency set for a previously announced command.
%Moreover, a dependency-set computation algorithm must ensure that:
%\begin{enumerate}[noitemsep,nolistsep]
  %  \item If the dependency set $d$ has been suggested for $c$, then every command in $\aset{c}\cup d$ must have been announced before.
  %  \item If the dependency set $d_1$ has been suggested for the command $c_1$, the dependency set $d_2$ has been suggested for the command $c_2$, and $c_1$ and $c_2$ do not commute, then either $c_1\in d_2$ or $c_2 \in d_1$.
%\end{enumerate}

%Any implementation of dependency-set algorithm can rely on the garbage collection of the map-agreement module to get rid of redundant information associated with specific commands in its data structures, with the purpose of preventing the dependency sets to grow indefinitely. 
 
%\todo{``Suggest'' can also be used to trigger recovery. Describe better this part.}
%\todo{Write a suggestion on about GC of dependencies. Remove any command c such that a committed command depends on c. Also us the GC of the map-agreement component.}

\subsection{Agreeing on Dependency Sets}\label{map-algo}
A map-agreement algorithm exposes the input interface $\opn{propose}_p\left( c, \mathcal{D} \right)$, to propose a set of dependencies $\mathcal{D}$ for a command $c$, and the output interface $\opn{commit}_p\left( c, \mathcal{D} \right)$, to commit $\mathcal{D}$ for $c$, where $p\in P$ is the process executing the call.
Moreover, a map-agreement algorithm must ensure that: 

\begin{enumerate}[noitemsep,nolistsep]
    \item[-] Safety: (\setword{S3}{commit-validity}), for any commit $\opn{commit}_p\left( c, \mathcal{D} \right)$, then $\mathcal{D}$ has been proposed for $c$ at an earlier time; (\setword{S4}{commit-agreement}), for any two decisions $\opn{commit}_p\left( c, \mathcal{D}_1 \right)$, $\opn{commit}_q\left( c, \mathcal{D}_2 \right)$, $\mathcal{D}_1$ is equal to $\mathcal{D}_2$.  
    \item[-] Liveness: (\setword{L3}{eventual-commit}) if a proposal $propose_p\left( c,\mathcal{D} \right)$ is made, then eventually a decision $\opn{commit}_p\left( c, \mathcal{D} \right)$ is made.
\end{enumerate}


\subsection{Local Dependency Graph Processing}
\label{abstract-algo}

The graph processing algorithm is executed locally by a process (line 10 of \cref{fig:deps-set-algo3}). It must compute the value of $\opn{c-struct}\left( g_p \right)$, as defined in \cref{lindef}.
Its key property is \cref{lemma:graph}. 
We assume that processes initially agree, for each set of commands $\mathcal{D}$, on a total order $<_{\mathcal{D}}$ on $\mathcal{D}$.
In practice, if commands are attached a unique identifier among a totally ordered set, $<_{\mathcal{D}}$ can be the restriction of this total order to $\mathcal{D}$.

We say that a command $c$ in the domain of $m_p$ is \emph{executable} when no command $c'$ which is not in the domain of $m_p$ can be reached from $c$ by recursively following $c$'s dependencies.
The local variable $m_p$ denotes a directed graph $g_p$ whose set of vertices $V\left( g_p \right)$ is the executable commands of $m_p$ and whose edges $E\left( g_p \right)$ are such that there is an edge from $c_1$ to $c_2$ iff $c_2 \in m_p\left[ c_1 \right]$ (i.e., $c_1$ depends on $c_2$). For example, if $m_p = \left[ c_1 \mapsto \aset{c_2,c_4}, c_2 \mapsto \aset{c_3}, c_3 \mapsto \aset{c_2}\right]$ then $V\left( g_p \right) = \aset{c_2,c_3}$ and $E\left( g_p \right) = \aset{\left( c_2,c_3 \right),\left( c_3,c_2 \right)}$.

A directed graph $g$ \emph{induces a partial order} $\preceq_g$ on its vertices defined such that $c_1 \preceq_g c_2$ iff there is a path from $c_2$ to $c_1$ and none from $c_1$ to $c_2$.
For example, consider $h$ where $V\left( h \right)=\aset{c_1,c_2,c_3,c_4}$ and $E\left( h \right)=\aset{\left( c_1,c_2 \right),\left( c_2,c_1 \right),\left( c_1,c_3 \right)}$. 
We have that $\preceq_h = \aset{\left( c_3,c_1 \right)}$.
We say that a total order $\ll$ on $V\left( g \right)$ is a \emph{linearization} of $g$ when for every $c_1,c_2\in V\left( g \right)$, $c_1 \preceq_g c_2$ implies $c_1 \ll c_2$, and if $c_1$ and $c_2$ belong to the same strongly connected component $\mathcal{D}$ and $ c_1 <_{\mathcal{D}} c_2$ hold, then $c_1 \ll c_2$. For example, assuming that $c_1 <_{\aset{c_1,c_2}} c_2$, the linearizations of $h$ are $\aseq{c_3,c_1,c_2,c_4}$, $\aseq{c_3,c_1,c_4,c_2}$, $\aseq{c_3,c_4,c_1,c_2}$,
$\aseq{c_4,c_3,c_1,c_2}$.

\begin{comment}
We now define a partial order on graphs. If $g_1$ and $g_2$ are two graphs, then we write $g_1 \leq g_2$ iff $V\left( g_1 \right)\subseteq V\left( g_2 \right)$, and if $v\in V\left( g_1 \right)$, $e\in E\left( g_2 \right)$, and $v$ is an endpoint of $e$, then $e\in E\left( g_1 \right)$. 
For example, consider the graphs $h'$ and $h''$ such that $V\left( h' \right)=\aset{c_1,c_2,c_4}$, $E\left( h' \right)=\aset{\left( c_1,c_2 \right),\left( c_2,c_1 \right)}$, $V\left( h'' \right)=\aset{c_1,c_3,c_4}$, and $E\left( h'' \right)=\aset{\left( c_1,c_3 \right)}$. We have that $h'\leq h$, but $ h''\not\leq h$.
\end{comment}

Define graph intersection such that $V\left( g_1 \cap g_2 \right)=V\left( g_1 \right)\cap V\left( g_2 \right)$ and $E\left( g_1 \cap g_2 \right)= E\left( g_1 \right)\cap E\left( g_2 \right)$, and graph union such that $V\left( g_1 \cup g_2 \right)=V\left( g_1 \right)\cup V\left( g_2 \right)$ and $E\left( g_1 \cup g_2 \right)= E\left( g_1 \right)\cup E\left( g_2 \right)$.
Define two graphs $g_1$ and $g_2$ as being \emph{compatible} iff $g_1 \cap g_2$ is a vertex-induced subgraph of $g_1 \cup g_2$ ($g$ is a vertex-induced subgraph of $g'$ iff $V\left( g \right)\subseteq V\left( g' \right)$, and for every $e\in E\left( g' \right)$, if both endpoints of $e$ are in $V\left( g \right)$, then $e\in E\left( g \right)$).

\begin{mylemma}\label{lemma:graph}
Assume that $g_1$ and $g_2$ are two dependency graphs, 
that for every pair of non-commutative commands $c_1,c_2 \in V\left(g_2\right)$, either $\left( c_1,c_2 \right)\in E\left( g_2 \right)$ or $\left( c_2,c_1 \right)\in E\left( g_2 \right)$,
that $g_1$ and $g_2$ are compatible,
that $l_1$ is a linearization of $g_1$, and that $l_2$ is a linearization of $g_2$. 
Then we have (a), $\bot\bullet l_1$ and $\bot\bullet l_2$ are compatible c-structs, and (b), if $g_1 = g_2$ then $\bot\bullet l_1 = \bot\bullet l_2$.
\end{mylemma}

\begin{mydef}\label{lindef}
We define $\opn{c-struct}\left( m_p \right) = \bot\bullet l$ where $l$ is a linearization of $g_p$.
By lemma 1.b, $\opn{c-struct}\left( m_p \right)$ is well-defined.
\end{mydef}

\subsection{Satisfying Consistency and Liveness of Generalized Consensus}
Consider an execution of the algorithm.
\begin{mylemma}\label{lemma:compat}
If $g_p^1$ is the graph $g_p$ at time $t_1$, $g_q^2$ is the graph $g_q$ at time $t_2$, then $g_p^1$ and $g_q^2$ are compatible.
\end{mylemma}
\begin{proof*}
TODO
%Follows from the safety property (b) of the map-agreement algorithm and definition of the graph $g_p$.
%From (b) it follows that if $c$ is in the domain of both $m_p$ and $m_q$ then $m_p\left[ c \right] = m_q\left[ c \right]$.
\end{proof*}
\begin{mythm}
    The consistency property of generalized consensus always holds.
    \label{lemma:consistency}
\end{mythm}
\begin{proof*}
    By \cref{lemma:compat} and \ref{suggestion-agreement}, $g_p^1$ and $g_q^2$ satisfy the premises of \cref{lemma:graph}. Therefore we get that $g_p^1$ and $g_q^2$ are compatible.
\end{proof*}
\begin{mythm}
    If a command $c$ which is GC-proposed is then repeatedly GC-proposed, then a c-struct containing $c$ is eventually decided.
\end{mythm}
\begin{proof*}
    The properties \ref{eventual-suggestion} and \ref{re-suggest} ensure that dependency sets are repeatedly proposed for $c$ to the map-agreement algorithm as long as a commit is not observed. Moreover, the map-agreement algorithm ensures that when dependency sets are repeatedly proposed for $c$, then a dependency set will eventually be committed for $c$. Similarly, any dependency of $c$ is eventually committed and $c$ becomes executable, at which points it is included in the next decided c-struct.
\end{proof*}


\begin{comment}
\textbf{Lemma 3}
Consider $g_1$ and $g_2$. Assume that if $v\in V\left( g_1 \right)$ and $v\in V\left( g_2 \right)$ then $n\left( g_1 \right)=n\left( g2 \right)$.

\textbf{Lemma 4}
If $m_p\subseteq m_q$

\end{comment}

\subsection{Implementing the Abstractions}
Both the dependency-set and map-agreement abstractions are well suited to leaderless implementations.

The dependency-set abstraction can be implemented as shown in \cref{fig:deps-set-algo1,fig:deps-set-algo2}. Due to space constraints, our example focuses on satisfying the safety property only. A process $p$ announcing a command $c$ asks all the processes for the set of commands they have seen so far and that do not commute with $c$. Then, it suggests the union of the dependency sets received from a \emph{quorum} of processes. Quorums are sets of processes such that the intersection of any two
quorums is not empty. Since $p$ receives sets of commands seen by other processes, the implementation trivially ensures \ref{dep-valid}. 
Further, consider two suggestions $\opn{suggest}\left(c_1,\matthcal{D}_1\right)$ and $\opn{suggest}\left(c_2,\matthcal{D}_2\right)$ where $c_1$ and $c_2$ do not commute. There are two quorums $Q_1$ and $Q_2$ such that $\matthcal{D}_1$ was computed from $Q_1$ and $\matthcal{D}_2$ was computed from $Q_2$. By the definition of quorums, there is a process $q$ belonging to both $Q_1$ and $Q_2$. This process $q$ received either $c_1$ before $c_2$, in which case $c_1\in
\mathcal{D}_2$, or $c_2$ before $c_1$, in which case $c_2\in\mathcal{D}_1$. Therefore, property \ref{suggestion-agreement} holds.

The map-agreement algorithm can be implemented by any state machine replication algorithm, like MultiPaxos~\cite{lamport2001paxos}, where a command $c$ is uniquely associated with a position in the sequence of consensus instances. %used to implementing map-agreement by uniquely associating commands to MultiPaxos instance numbers.
However, MultiPaxos is too strong because we do not care about the ordering among instances.
Instead, one can uniquely associate one incarnation of MultiPaxos with each process, where each process is initially the leader of its MultiPaxos incarnation.  
We assume that a command is associated with a unique natural number $\opn{id}\left( c \right)$ and a process id $\opn{pid}\left( c \right)$, corresponding to the first process that received $c$.
A process receiving a proposal $\opn{propose}_p\left( c, \mathcal{D}) $  proposes $ c\mapsto\mathcal{D} $ in instance $\opn{id}\left( c \right)$ of the MultiPaxos incarnation of process $\opn{pid}\left( c \right)$. Upon a MultiPaxos decision $ c\mapsto\mathcal{D'} $, $p$ calls $\opn{commit}_p\left( c,\mathcal{D} \right)$.
Note that in a failure-free case, a command is committed in two message delays by its originator.

EPaxos improves upon our MultiPaxos scheme in two ways. First, EPaxos uses Fast Paxos\cite{Lamport05GeneralizeConsensus} incarnations instead of MultiPaxos incarnations. Second, EPaxos combines the dependency-set algorithm outlined above, but using fast quorums, with the first phase of Fast Paxos: the state of the dependency-set algorithm and of the Fast Paxos incarnations are merged so as to allow the reception of identical dependency sets from a fast quorum of processes
(after an $\opn{announce}_p\left( c \right)$) to act as a fast-path decision of Fast Paxos. 


%A process receiving a proposal $ c\mapsto\mathcal{D} $ attaches a unique natural number $\opn{id}\left( c \right)$ as command identifier and its own id $\opn{pid}\left( c \right)$. proposes $ c\mapsto\mathcal{D} $ in instance $\opn{id}\left( c \right)$ of its MultiPaxos incarnation. Upon a MultiPaxos decision $ c\mapsto\mathcal{D'} $, $p$ commits $
%c\mapsto\mathcal{D'} $. 

\printbibliography%
\begin{comment}
\appendix

\section{Generalized Consensus}

A c-struct set is a set containing the element $\bot$ and with an append operator $\bullet$ such that $\sigma \bullet c$ is a c-struct, for any c-struct $\sigma$ and command $c$. 
We also write the $\sigma\bullet\aseq{c_1,\cdots,c_n}$ for the c-struct $\sigma\bullet c_1\bullet\cdots\bullet c_n$.
C-structs are partially ordered by the relation $\sqsubseteq$ defined such that $\sigma_1\sqsubseteq\sigma_2$ iff there exists a sequence of commands $\mathcal{D}$ such that $\sigma_2=\sigma_1\bullet \mathcal{D}$. Two c-structs are compatible when they have a common upper bound; when this is the case, two c-structs $\sigma_1$ and $\sigma_2$ have a least upper bound, noted $\bigsqcup\left( \sigma_1,\sigma_2 \right)$.
We say that two commands \textit{commute} when $\sigma\bullet c_1\bullet c_2 =  \sigma\bullet c_2\bullet c_1 $, for every c-struct $\sigma$. Intuitively, a c-struct represents a set of sequences that are all equivalent up to the ordering of commutative commands (see~\cite{Lamport05GeneralizeConsensus} for a complete discussion on c-struct). 
In the rest of the paper the symbol $c$, possibly sub-scripted, ranges over commands, and the symbol $\mathcal{D}$, possibly sub-scripted, ranges over sets of commands.

We can now state the generalized-consensus properties: non-triviality requires that any decided c-struct be of the form $\bot\bullet \mathcal{D}$, where $\mathcal{D}$ is a sequence of proposed commands; consistency requires than any two decided c-structs $\sigma_1$ and $\sigma_2$ be compatible; stability requires that when a process $p$ decides a c-struct $\sigma_1$ and time $t_1$ and $\sigma_2$ at time $t_2$, then $t_1 \leq t_2$ implies that $\sigma_1\sqsubseteq \sigma_2$; finally, liveness requires that
if a command keeps beeing proposed, then a c-struct $\sigma$ containing the command is eventually decided.

\subsection{Description map-agreement implementation}

An example of implementation of such an algorithm is depicted by the pseudocode of Figures~\ref{fig:deps-set-algo1} and~\ref{fig:deps-set-algo2}. The idea is to let every process broadcast $c$ to all the processes in the system, in order to announce $c$ (lines \ref{alg:announce}--\ref{alg:send-new-cmd}). Then whenever a process $p$ receives $c$, it replies to the sender with the set $\mathcal{D}$ of commands that $p$ has observed so far, and such that $c$ and any command in $\mathcal{D}$ do not commute (lines \ref{alg:receive-new-cmd}--\ref{alg:send-deps}). Afterwards, the process announcing $c$ collects the sets $\mathcal{D}$ for $c$ from a \textit{quorum} of $\mathcal{Q}$ processes, and it locally suggests the union of those sets as final dependency-set for $c$ (lines \ref{alg:receive-deps}--\ref{alg:call-suggest}). For the purpose of this example we allow for $\mathcal{Q}$ to be equal to any positive integer that is not greater than the number of processes, as long as the following requirement is met: the intersection of any two quorums that are selected by processes to collect the sets of dependencies is not empty.

The provided implementation satisfies the specification of a dependency-set algorithm for the following reasons. Since any command in a dependency-set that is suggested by a process (lines~\ref{alg:update-deps-p} and~\ref{alg:call-suggest}) must be in the set $cmds_q$ of the commands received by some process $q$ (line~\ref{alg:update-d}), and a process receives a command iff there exists some process that previously announced the command (line~\ref{alg:send-new-cmd}), then it follows that any command in any suggested dependency-set has been previously announced. Further, since a process can suggest a set of dependencies $deps_p[c]$ for a command $c$ after it has received a reply for $c$ from at least a quorum of processes (lines~\ref{alg:receive-deps}--\ref{alg:call-suggest}), we have that $\opn{suggest}_p \left( c,deps_p[c]\right)$ is called only if a quorum of processes $q$ have $c$ in their $cmds_q$ (line~\ref{alg:update-cmds-p}). Therefore, by the requirement that we enforced on the quorums, we can conclude that for any two suggestions $\opn{suggest}_p \left( c_1,deps_p[c_1]\right)$ and $\opn{suggest}_q \left( c_2,deps_q[c_2]\right)$,  if $c_1$ and $c_2$ do not commute then either $c_1\in deps_q[c_2]$ or $c_2 \in deps_p[c_1]$.

\end{comment}
\end{document}
